\documentclass[9pt]{beamer}
\usetheme{metropolis}

\makeatletter
\setbeamertemplate{headline}{%
  \begin{beamercolorbox}[colsep=1.5pt]{upper separation line head}
  \end{beamercolorbox}
  \begin{beamercolorbox}{section in head/foot}
    \vskip2pt\insertnavigation{\paperwidth}\vskip2pt
  \end{beamercolorbox}%
  \begin{beamercolorbox}[colsep=1.5pt]{lower separation line head}
  \end{beamercolorbox}
}
\makeatother

\setbeamercolor{section in head/foot}{fg=normal text.bg, bg=structure.fg}


\usepackage{booktabs}
\usepackage{longtable}
\usepackage{array}
\usepackage{multirow}
\usepackage{wrapfig}
\usepackage{float}
\usepackage{colortbl}
\usepackage{pdflscape}
\usepackage{tabu}
\usepackage{threeparttable}
\usepackage{threeparttablex}
\usepackage[normalem]{ulem}
\usepackage{makecell}
\usepackage{xcolor}
\usepackage[utf8]{inputenc}
\usepackage[absolute,overlay]{textpos}

\definecolor{red}{RGB}{150, 0, 0}
\definecolor{RED}{RGB}{255, 0, 0}
\definecolor{green}{RGB}{0, 85, 0}

\definecolor{GreenTOL}{HTML}{225522}
\setbeamercolor{example text}{fg=GreenTOL}
\setbeamercolor{alerted text}{fg=red}
\metroset{block=fill}
\setbeamercolor{block title alerted}{use=alerted text, fg=alerted text.fg, bg=alerted text.bg!80!alerted text.fg}
\setbeamercolor{block body alerted}{use={block title alerted, alerted text}, fg=alerted text.fg, bg=block title alerted.bg!50!alerted text.bg}
\setbeamercolor{block title example}{use=example text, fg=example text.fg, bg=example text.bg!80!example text.fg}
\setbeamercolor{block body example}{use={block title example, example text}, fg=example text.fg, bg=block title example.bg!50!example text.bg}

\title{A log-ratio approach to cluster analysis of count data when the total is irrelevant}
\date{}
\author{%
  \texorpdfstring{
    \begin{columns}%[onlytextwidth]
      \column{.35\linewidth}
      \centering
      \textbf{M. Comas-Cufí\inst{1}}\\
      \href{mailto:marc.comas@udg.edu}{marc.comas@udg.edu}\\ \vspace{0.5cm}
      \
      G. Mateu-Figueras\inst{1}\\
      \href{mailto:gloria.mateu@udg.edu}{gloria.mateu@udg.edu}\\
      \column{.35\linewidth}
      \centering
      J.A. Martín-Fernández\inst{1}\\
      \href{mailto:josepantoni.martin@udg.edu}{josepantoni.martin@udg.edu}\\   \vspace{0.5cm}
      J. Palarea-Albaladejo\inst{2}\\
      \href{mailto:javier.palarea@bioss.ac.uk}{javier.palarea@bioss.ac.uk}      
    \end{columns}
  }
  {}
}

\institute{\vspace{0.25cm}
\inst{1} Department of Computer Science, Applied Mathematics and Statistics, Universitat de Girona, Girona \\
\includegraphics[height=0.9cm]{imae.png}
\and%
\inst{2} Biomathematics and Statistics Scotland, Edinburgh\\
\includegraphics[height=1cm]{bioss.png}}

    
\begin{document}
\begin{frame}[noframenumbering]
\thispagestyle{empty}
\titlepage
\end{frame}

{
\metroset{sectionpage=none}
\section{Preliminaries}
}

\begin{frame}{Compositional data analysis}

\begin{itemize}
\item
  \textbf{Compositional data} (CoDa), \((p_1, \dots, p_{{D}})\), are
  quantitative descriptions of the parts of some whole, conveying
  \textbf{relative information}. CoDa are commonly expressed in
  proportions, percentages, or ppm  (Aitchison, 1986).
\item
The simplex
\[
S^{{D}} = \left\{ \left(p_1,\dots,p_{{D}}\right) | \; p_i > 0, \sum_{i=1}^{{D}} p_i = \kappa \right\},
\] is the sample space of CoDa.
\item
  \textbf{Log-ratios} of parts handle relative information and satisfy desirable properties such as scale invariance and subcompositional coherence: \[
  \log\left(\frac{p_i}{p_j}\right), \log\left(\frac{p_j}{\sqrt[{D}]{\prod_{\ell=1}^{{D}}p_{\ell}}} \right), \sqrt{\frac{j}{j+1}}\, \log\frac{\sqrt[j]{\prod_{\ell=1}^{j} p_\ell}}{p_{j+1}}, \dots
  \]
\end{itemize}

\end{frame}

\begin{frame}{CoDa and the zero problem}

\begin{itemize}
\item Zeros prevent from using log-ratios. Most proposals have been focused on the continuous case (\textbf{zCompositions} R package; Palarea-Albaladejo \& Martín-Fernández, 2015).
\item
  \textbf{Compositional count data sets:} discrete vectors of number of outcomes falling into mutually exclusive categories.
\begin{table}[ht]
\centering
\scriptsize
\begin{tabular}{lrrrrrrrrrr}
  \hline
\textbf{Municipality} & \textbf{jxsi} & \textbf{psc} & \textbf{pp} & \textbf{catsp} & \textbf{cs} & \textbf{cup}  \\ 
  \hline
   S. Jaume de F. &  14 &   1 &   {\color{red}0} &   2 &   {\color{red}0} &   5  \\ 
  Gisclareny &  20 &   {\color{red}0} &   {\color{red}0} &   {\color{red}0} &   1 &   2  \\ 
\vdots &  \vdots &  \vdots & \vdots & \vdots  &  \vdots & \vdots  \\
L'Hosp. de Llob. & 23843 & 28947 & 14336 & 16855 & 29773 & 7528\\
Barcelona & 326376 & 100806 & 80529 & 85841 & 155361 & 87774 \\
\end{tabular}
\end{table}
\item \textbf{Assumption 1:} The relative information is relevant, the total is not.  
\item \textbf{Assumption 2:} The probability of a part different of zero is not zero. Zeros due to sampling limitations.

\end{itemize}

\end{frame}

\begin{frame}{Parametric approaches to cluster compositional count data set (1)}

\begin{itemize}
\item \textbf{{\color{red}Compositional} and {\color{red}count} variability not taken into account}.\vspace{0.5cm}
\item \textbf{{\color{green}Count} variability taken into account, but not {\color{red}compositional} variability}. Mixtures of multinomial distributions.\vspace{0.5cm}
\item \textbf{{\color{green}Compositional} variability taken into account, but not {\color{red}count} variability}. Zero multiplicative replacement methods.
\begin{itemize}
\item Dirichlet prior  (Martín-Fernández \textit{et al.}, 2015)
\item Log-ratio normal  prior (Comas-Cufí \textit{et al.}, 2019)
\end{itemize}
\end{itemize}

\end{frame}

\begin{frame}{Parametric approaches to cluster compositional count data set (2)}


\begin{itemize}
\item \textbf{{\color{green}Compositional} and {\color{green}count} variability taken into account}.\vspace{0.25cm}
\begin{itemize}
\item \textit{Topic models}. Mixture of multinomials where mixing proportions are modelled in the Simplex.\vspace{0.25cm}
\begin{itemize}
\item Latent Dirichlet Allocations (Blei \textit{et al.} 2003)
\item Correlated Topic Models (Blei \& Lafferty, 2007)
\end{itemize}\vspace{0.25cm}
\item \textit{Mixtures of compounding distributions}.\vspace{0.25cm}
\begin{itemize}
\item Mixtures of Dirichlet-multinomial distributions (Holmes et al., 2012)
\item Mixtures of log-ratio-normal-multinomial distributions (Comas-Cufí et al., 2017)
\end{itemize}
\end{itemize}
\end{itemize}

\pause
\begin{alertblock}{Main limitations}
\begin{itemize}
\item Dirichlet-based approaches have  \textbf{modelling issues}.
\item Gaussian-based approaches have  \textbf{estimation issues}.
\end{itemize}
\end{alertblock}

\end{frame}

\section{Our proposal}

\begin{frame}{Classical clustering approaches applied to cluster count data}

\begin{enumerate}
\item \textbf{Dealing with zeros}. 
\begin{itemize}
\item Expected values of a Dirichlet-multinomial (DM) distribution seems to be conservative in keeping the covariance structure observed in the original count data set. The regression toward the mean is moderate.
\item Zero replacement is even more conservative in keeping covariance structure observed in the count data set. But counts with small parts tend to define clusters by themselves.
\end{itemize}
\item \textbf{Compositional variability}. Model your data using a generic distribution defined on the Simplex. Gaussian mixtures  are easy to estimate.
\item \textbf{Count variability}. Create $B$ new samples using the posterior distribution, and find clusters using classical methods on its log-ratio coordinates.
\item \textbf{Consensus clustering}. Use a clustering ensemble method to build a final cluster  (e.g. majority voting (Dudoit \& Fridlyand, 2003)).
\end{enumerate}
\end{frame}

\section{Example: 2017 Catalan regional election}

\begin{frame}{Multivariate count data set}

\begin{itemize}
\item 947 municipalities. To illustrate the approach we only consider three parts obtained with the following amalgamations.
\begin{itemize}
\item \textbf{Pro-independence parties (ind)}: CUP (cup), Esquerra Repúblicana de Catalunya (erc), Junts per Catalunya (jxcat).
\item \textbf{Anti-independence parties (esp)}: Ciutadans (cs), Partit Popular (pp), Partit Socialista de Catalunya (psc).
\item \textbf{Mixed opinions (other)}: Catalunya si que es pot (catsp), others (other).
\end{itemize}

\end{itemize}

\only<1>{\input{table01a.tex}}
\only<2>{\begin{table}[H]
\centering\begingroup\fontsize{7}{9}\selectfont

\begin{tabular}{lrrrlrrrlrrrlrrr}
\toprule
mun & ind & esp & other\\
\midrule
Abella de la Conca & 88 & 22 & 4\\
Abrera & 2243 & 4367 & 973\\
Agramunt & 2204 & 758 & 87\\
Aguilar de Segarra & 160 & 24 & 1\\
$\vdots$ & $\vdots$ & $\vdots$ & $\vdots$\\
\bottomrule
\end{tabular}
\endgroup{}
\end{table}
}


\end{frame}


\begin{frame}{Dealing with zeros}

\begin{columns}
\begin{column}{0.5\textwidth}
\begin{figure}\vspace{-0.20cm}
\includegraphics[trim=0cm 0cm 0cm 0cm,width=1.2\textwidth]{ternary_original.pdf}
\end{figure}
\end{column}
\begin{column}{0.5\textwidth}
\begin{itemize}
\item Most municipalities lie between \textbf{ind} and \textbf{esp} parties.
\item Some municipalities have some zero (see ${\color{RED}\bullet}$).
\item<2>[$\rightarrow$] \textbf{We will deal with zeros first.}
\end{itemize}
\end{column}
\end{columns}

\end{frame}



\begin{frame}{Dealing with zeros}

Here, we consider two different approaches: \begin{itemize}\item Geometric Bayesian multiplicative (Martín-Fernández, 2015), and \item Dirichlet-multinomial smoothing after replacing by the expected posterior probabilities.\end{itemize}%

\begin{columns}
\begin{column}{0.45\textwidth}
\begin{figure}\vspace{-0.20cm}
\includegraphics[trim=0cm 0cm 0cm 0cm,width=\textwidth]{ternary_zr.pdf}
\end{figure}
\end{column}
\begin{column}{0.45\textwidth}
\begin{figure}\vspace{-0.20cm}
\includegraphics[trim=0cm 0cm 0cm 0cm,width=\textwidth]{ternary_nz.pdf}
\end{figure}
\end{column}
\end{columns}

\end{frame}

\begin{frame}[t]{Dealing with zeros}

\begin{columns}
\begin{column}{0.5\textwidth}
\begin{figure}\vspace{-0.20cm}
\includegraphics[trim=0cm 0cm 0cm 0cm,width=\textwidth]{coordinates_zr.pdf}
\end{figure}
\end{column}
\begin{column}{0.5\textwidth}
\begin{figure}\vspace{-0.20cm}
\includegraphics[trim=0cm 0cm 0cm 0cm,width=\textwidth]{coordinates_nz.pdf}
\end{figure}
\end{column}
\end{columns}

\begin{table}[H]
\centering\begingroup\fontsize{7}{9}\selectfont

\begin{tabular}{lrrr}
\toprule
Basis $\mathcal{B}$ & ind & esp & other\\
\midrule
B1 & 1 & -1 & 0\\
B2 & 1 & 1 & -1\\
\bottomrule
\end{tabular}
\endgroup{}
\end{table}


\end{frame}

\begin{frame}[t]{Clustering directly in count data}

\begin{columns}
\begin{column}{0.5\textwidth}
\begin{figure}\vspace{-0.20cm}
\only<1>{\includegraphics[trim=0cm 0cm 0cm 0cm,width=\textwidth]{coordinates_black_zr.pdf}}%
\only<2-3>{\includegraphics[trim=0cm 0cm 0cm 0cm,width=\textwidth]{clustering0_zr.pdf}}
\end{figure}
\end{column}
\begin{column}{0.5\textwidth}
\begin{figure}\vspace{-0.20cm}
\only<1>{\includegraphics[trim=0cm 0cm 0cm 0cm,width=\textwidth]{coordinates_black_nz.pdf}}%
\only<2-3>{\includegraphics[trim=0cm 0cm 0cm 0cm,width=\textwidth]{clustering0_nz.pdf}}
\end{figure}
\end{column}
\end{columns}

\vspace{-0.26cm}
\only<1-2>{We can cluster our compositional data for example using $k$-means. \begin{itemize}\item  Duda-Hart test was used to discard one cluster. \item Calinski-Harabasz index was used to select $k$ between 2 and 10. \end{itemize}}
\only<3>{\begin{alertblock}{Limitations}\begin{itemize}\item In the zero-replacement approach, observations with a small amount of counts tend to create clusters.\item In DM smoothing results can be affected by the Dirichlet prior.\end{itemize}\end{alertblock}}
\end{frame}

\begin{frame}[t]{Compositional variability}

\begin{columns}
\begin{column}{0.5\textwidth}
\begin{figure}\vspace{-0.20cm}%
\only<1>{\includegraphics[trim=0cm 0cm 0cm 0cm,width=\textwidth]{coordinates_black_zr.pdf}}%
\only<2>{\includegraphics[trim=0cm 0cm 0cm 0cm,width=\textwidth]{model_zr.pdf}}%
\end{figure}
\end{column}
\begin{column}{0.5\textwidth}%
\begin{figure}\vspace{-0.20cm}
\only<1>{\includegraphics[trim=0cm 0cm 0cm 0cm,width=\textwidth]{coordinates_black_nz.pdf}}%
\only<2>{\includegraphics[trim=0cm 0cm 0cm 0cm,width=\textwidth]{model_nz.pdf}}%
\end{figure}
\end{column}
\end{columns}

\vspace{-0.26cm}
\only<1-2>{\begin{exampleblock}{Modelling using Gaussian mixtures}\vspace{0.1cm}Find a distribution to model the original sample. Mixtures of Gaussian distribution are a good option (Nguyen \& McLachlan, 2018)\begin{itemize}\item[$\rightarrow$] We estimate a mixture of \emph{ten} Gaussian distributions with equal volume.\end{itemize}\end{exampleblock}}

\end{frame}



\begin{frame}[t]{Count variability}

\begin{columns}
\begin{column}{0.5\textwidth}
\begin{figure}\vspace{-0.20cm}
\only<1>{\includegraphics[trim=0cm 0cm 0cm 0cm,width=\textwidth]{model_zr.pdf}}%
\only<2>{\includegraphics[trim=0cm 0cm 0cm 0cm,width=\textwidth]{posterior_zr_60.pdf}}%
\only<3>{\includegraphics[trim=0cm 0cm 0cm 0cm,width=\textwidth]{posterior_zr_66.pdf}}%
\only<4>{\includegraphics[trim=0cm 0cm 0cm 0cm,width=\textwidth]{posterior_zr_90.pdf}}%
\end{figure}
\end{column}
\begin{column}{0.5\textwidth}
\begin{figure}\vspace{-0.20cm}
\only<1>{\includegraphics[trim=0cm 0cm 0cm 0cm,width=\textwidth]{model_nz.pdf}}%
\only<2>{\includegraphics[trim=0cm 0cm 0cm 0cm,width=\textwidth]{posterior_nz_60.pdf}}%
\only<3>{\includegraphics[trim=0cm 0cm 0cm 0cm,width=\textwidth]{posterior_nz_66.pdf}}%
\only<4>{\includegraphics[trim=0cm 0cm 0cm 0cm,width=\textwidth]{posterior_nz_90.pdf}}%
\end{figure}
\end{column}
\end{columns}

\vspace{-0.2cm}
\begin{exampleblock}{Sampling from the posterior distribution}
For each count observation $\textbf{x}_i$, we sample from the posterior distribution $P(\textbf{h} \mid \textbf{x}_i; f)$, where $f$ is the mixture of Gaussian distributions.
\begin{itemize}
\item Metropolis–Hastings using $g(\textbf{h}) = f(\textbf{h})\cdot\text{Mult}(\textbf{x}_i; \textbf{ilr}_\mathcal{B}^{-1}(\textbf{h}))$, and proposal step using the Laplace approximation of $P(\textbf{h} \mid \textbf{x}_i; f)$ centred at zero.
\uncover<2->{
\only<1-2>{\item{$i=\text{``Argelaguer''},\; \textbf{x}_i=(\text{ind: } 259, \text{esp: } 19, \text{other: } 14)$}}%
\only<3>{\item{$i=\text{``Arres''},\; \textbf{x}_i=(\text{ind: } 10, \text{esp: } 28, \text{other: } 0)$}}%
\only<4>{\item{$i=\text{``Barcelona''},\; \textbf{x}_i=(\text{ind: } 429\,782, \text{esp: } 405\,924, \text{other: } 96\,748)$}}%
}
\end{itemize}
\end{exampleblock}


\end{frame}


\begin{frame}[t]{Compositional and count variability}

\begin{columns}
\begin{column}{0.5\textwidth}
\begin{figure}\vspace{-0.20cm}%
\only<1>{\includegraphics[trim=0cm 0cm 0cm 0cm,width=\textwidth]{model_zr.pdf}}%
\only<2>{\includegraphics[trim=0cm 0cm 0cm 0cm,width=\textwidth]{sample_zr_1.pdf}}%
\only<3>{\includegraphics[trim=0cm 0cm 0cm 0cm,width=\textwidth]{sample_cl_zr_1.pdf}}%
\only<4>{\includegraphics[trim=0cm 0cm 0cm 0cm,width=\textwidth]{sample_zr_2.pdf}}%
\only<5>{\includegraphics[trim=0cm 0cm 0cm 0cm,width=\textwidth]{sample_cl_zr_2.pdf}}%
\only<6>{\includegraphics[trim=0cm 0cm 0cm 0cm,width=\textwidth]{sample_zr_3.pdf}}%
\only<7>{\includegraphics[trim=0cm 0cm 0cm 0cm,width=\textwidth]{sample_cl_zr_3.pdf}}%
\only<8>{\includegraphics[trim=0cm 0cm 0cm 0cm,width=\textwidth]{sample_zr_4.pdf}}%
\only<9>{\includegraphics[trim=0cm 0cm 0cm 0cm,width=\textwidth]{sample_cl_zr_4.pdf}}%
\end{figure}
\end{column}
\begin{column}{0.5\textwidth}%
\begin{figure}\vspace{-0.20cm}
\only<1>{\includegraphics[trim=0cm 0cm 0cm 0cm,width=\textwidth]{model_nz.pdf}}%
\only<2>{\includegraphics[trim=0cm 0cm 0cm 0cm,width=\textwidth]{sample_nz_1.pdf}}%
\only<3>{\includegraphics[trim=0cm 0cm 0cm 0cm,width=\textwidth]{sample_cl_nz_1.pdf}}%
\only<4>{\includegraphics[trim=0cm 0cm 0cm 0cm,width=\textwidth]{sample_nz_2.pdf}}%
\only<5>{\includegraphics[trim=0cm 0cm 0cm 0cm,width=\textwidth]{sample_cl_nz_2.pdf}}%
\only<6>{\includegraphics[trim=0cm 0cm 0cm 0cm,width=\textwidth]{sample_nz_3.pdf}}%
\only<7>{\includegraphics[trim=0cm 0cm 0cm 0cm,width=\textwidth]{sample_cl_nz_3.pdf}}%
\only<8>{\includegraphics[trim=0cm 0cm 0cm 0cm,width=\textwidth]{sample_nz_4.pdf}}%
\only<9>{\includegraphics[trim=0cm 0cm 0cm 0cm,width=\textwidth]{sample_cl_nz_4.pdf}}%
\end{figure}
\end{column}
\end{columns}

\vspace{-0.26cm}
\begin{exampleblock}{Creating new samples}
\begin{itemize}
\item[$\rightarrow$]<1-> Using the posterior distribution we can create  $B$  new samples ($B=100$),
\item[$\rightarrow$]<2-> and applying a clustering algorithm ($k$-means, $k\in\{2,\dots,10\}$, Calinski-Harabasz index).
\end{itemize}
\end{exampleblock}

\end{frame}

\begin{frame}[t]{Consensus clustering}

\begin{itemize}
\item Consensus clustering can be applied to summarise all B clusterings. 
\begin{itemize}\item[$\rightarrow$] Majority voting (Dudoit and Fridlyand, 2003)).\end{itemize}
\item<4> Only \emph{six} municipalities differ in the final clustering.
\end{itemize}

\begin{columns}
\begin{column}{0.5\textwidth}
\begin{figure}\vspace{-0.20cm}%
\only<1>{\includegraphics[trim=0cm 0cm 0cm 0cm,width=0.45\textwidth]{sample_cl_zr_1.pdf}\includegraphics[trim=0cm 0cm 0cm 0cm,width=0.45\textwidth]{sample_cl_zr_2.pdf}\\\includegraphics[trim=0cm 0cm 0cm 0cm,width=0.45\textwidth]{sample_cl_zr_3.pdf}\includegraphics[trim=0cm 0cm 0cm 0cm,width=0.45\textwidth]{sample_cl_zr_4.pdf}}
%\only<10>{\includegraphics[trim=0cm 0cm 0cm 0cm,width=\textwidth]{sample_zr_5.pdf}}%
%\only<11-12>{\includegraphics[trim=0cm 0cm 0cm 0cm,width=\textwidth]{sample_cl_zr_5.pdf}}%
\only<2>{\includegraphics[trim=0cm 0cm 0cm 0cm,width=\textwidth]{clustering_zr.pdf}}%
\only<3>{\includegraphics[trim=0cm 0cm 0cm 0cm,width=\textwidth]{clustering_tern_zr.pdf}}%
\only<4>{\includegraphics[trim=0cm 0cm 0cm 0cm,width=\textwidth]{clustering_tern_diff_zr.pdf}}%
\end{figure}
\end{column}
\begin{column}{0.5\textwidth}%
\begin{figure}\vspace{-0.20cm}
\only<1>{\includegraphics[trim=0cm 0cm 0cm 0cm,width=0.45\textwidth]{sample_cl_nz_1.pdf}\includegraphics[trim=0cm 0cm 0cm 0cm,width=0.45\textwidth]{sample_cl_nz_2.pdf}\\\includegraphics[trim=0cm 0cm 0cm 0cm,width=0.45\textwidth]{sample_cl_nz_3.pdf}\includegraphics[trim=0cm 0cm 0cm 0cm,width=0.45\textwidth]{sample_cl_nz_4.pdf}}%
%\only<10>{\includegraphics[trim=0cm 0cm 0cm 0cm,width=\textwidth]{sample_nz_5.pdf}}%
%\only<11-12>{\includegraphics[trim=0cm 0cm 0cm 0cm,width=\textwidth]{sample_cl_nz_5.pdf}}%
\only<2>{\includegraphics[trim=0cm 0cm 0cm 0cm,width=\textwidth]{clustering_nz.pdf}}%
\only<3>{\includegraphics[trim=0cm 0cm 0cm 0cm,width=\textwidth]{clustering_tern_nz.pdf}}%
\only<4>{\includegraphics[trim=0cm 0cm 0cm 0cm,width=\textwidth]{clustering_tern_diff_nz.pdf}}%
\end{figure}
\end{column}
\end{columns}

\end{frame}


\begin{frame}{Final remarks}

\begin{itemize}
\item An approach to cluster count data when only the relative relation between parts is of interest has been presented.\vspace{0.2cm}
%\item Methods using compositional covariance (those based on the normality) have very limited applicability. They are computational demanding.\vspace{0.2cm}
\item A parametric approach can be constructed in such a way that the variability coming from a multinomial counting process can be incorporated to the observed compositional variability.\vspace{0.2cm}
\item To obtain a final clustering, consensus clustering algorithms can be applied to clustering obtained  from each sample.
\end{itemize}
\end{frame}






\end{document}
